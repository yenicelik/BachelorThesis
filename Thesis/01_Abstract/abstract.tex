% ************************** Thesis Abstract *****************************
% Use `abstract' as an option in the document class to print only the titlepage and the abstract.
\begin{abstract}

In this thesis, I explore the possibilities of conducting Bayesian optimization techniques in high dimensional domains.
Although high dimensional domains can be defined to be between hundreds and thousands of dimensions, we will primarily focus on problem settings that occur between two and 20 dimensions.
As such, we focus on solutions to practical problems, such as tuning the parameters for an electron accelerator, or for even simpler tasks that can be run and optimized just in time with a standard laptop at hand. \\

Our main contributions is 1.) comparing how the log-likelihood affects the angle-difference in the real projection matrix, and the found matrix matrix, 2.) an extensive analysis of current popular methods including strengths and shortcomings, 3.) a short analysis on how dimensionality reduction techniques can be used for feature selection, and 4.) a novel algorithm called "BORING", which allows for a simple fallback mechanism if the matrix identification fails, as well as taking into consideration "passive" subspaces which provide small perturbations of the function at hand. \\

The main features of BORING are 1.) the possibility to identify the subspace (unlike most other optimization algorithms), and 2.) to provide a much lower penalty to identify the subspace if identification fails, as optimization is still the primary goal.

\end{abstract}
