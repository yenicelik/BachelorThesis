%!TEX root = ../thesis.tex
%*******************************************************************************
%****************************** Third Chapter **********************************
%*******************************************************************************
\chapter{Implementing methods}

% **************************** Define Graphics Path **************************
\ifpdf
    \graphicspath{{Chapter3/Figs/Raster/}{Chapter3/Figs/PDF/}{Chapter3/Figs/}}
\else
    \graphicspath{{Chapter3/Figs/Vector/}{Chapter3/Figs/}}
\fi

\section{Difficulties when implementing models}
When implementing models, the following difficulties may arise:

\begin{itemize}
\item Overflows and underflow are often occuring as a reusult of high dimesionality
\item Making sure the numerical examples are correct
\item Making sure the differentiations are correct (use finite differences for matrices in this example)
\item Write theoretical tests of the entire results
\item test individual parts
\item Test if the expectation $$ E[ f(A x) - f_hat(A_hat x) ] $$ approaches zero
\item Visualize the individual functions
\item write a very simple test case of a parabola, and check where the points project to
\item Performance of the implementation might not be optimal, as the design is only clear after implementation
\end{itemize}

\section{Family of matrices}
It turns out that there is a family of matrices that all project to the same space.
As such, we need better measures to 
\begin{itemize}
\item choose the matrix that we will take amongst all initialized tries (should be in the paper)
\item 
\end{itemize}

\section{Methods from }
And now I begin my third chapter here \dots

And now to cite some more people~\citet{Rea85,Ancey1996}

\subsection{First subsection in the first section}
\dots and some more 


