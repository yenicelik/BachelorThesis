%!TEX root = ../thesis.tex
%*******************************************************************************
%****************************** Third Chapter **********************************
%*******************************************************************************
\chapter{Conclusion}

% **************************** Define Graphics Path **************************
\ifpdf
    \graphicspath{{Chapter7/Figs/Raster/}{Chapter7/Figs/PDF/}{Chapter7/Figs/}}
\else
    \graphicspath{{Chapter7/Figs/Vector/}{Chapter7/Figs/}}
\fi

\section{Main Contributions}
I present this thesis as a result of a systematic treatment of the topic "Bayesian Optimization in High Dimensions".

\begin{enumerate}
\item BORING allows the number of dimensions to be smaller than the real active subspace.
Additional vector elements can be added at a very small cost, keeping the major principal axis untouched.
\item BORING is more robust to smaller perturbations, as it does not rely on finding the exact active subspace (which can be ambigious if we want the "major" active subspace).
It allows to optimize over smaller perturbations aswell, whereas current algorithms usually don't take these into consideration.
\end{enumerate}

After doing a throughout literature review, I was able to identify shortcomings of current methods.
Together with my supervisors, were able to come up with the new algorithm "BORING" that improves on the state-of-the-art for Bayesian Optimization algorithms that are able to identify the subspace projection matrix as part of the optimization procedure.
In contrast to other algorithms that solely focus on finding a projection, "BORING" takes into consideration smaller perturbations, and makes the GP process more robust by allowing addition projection matrices to optimize over other axis of the subspace. \\

\section{Conclusion}

Our empirical results show that "BORING" provides a new state-of-the-art for Bayesian Optimization algorithms that use subspace projection as the main feature.
This allows for interpretable future work
 



