%!TEX root = ../thesis.tex
%*******************************************************************************
%****************************** Third Chapter **********************************
%*******************************************************************************
\chapter{Evaluation}

% **************************** Define Graphics Path **************************
\ifpdf
    \graphicspath{{Chapter6/Figs/Raster/}{Chapter6/Figs/PDF/}{Chapter6/Figs/}}
\else
    \graphicspath{{Chapter6/Figs/Vector/}{Chapter6/Figs/}}
\fi

\section{Evaluation Settings}

Appendix A presents a list of synthetic functions and real datasets that are used to evaluate the effectiveness of a bayesian optimization algorithm. 
We will briefly go over some synthetic functions that we will be using for the evaluation of our algorithm.
Furthermore, we will shortly discuss the underlying real dataset, which is hyper-parameter-configurations from the SwissFEL x-ray laser.

\section{Quantitative evaluation}

We will conduct experiments in the following settings (as mentioned in some other chapter REEEE):

\begin{enumerate}
\item a 2D embedded function in a 3D (synthetic).
\item a 2D embedded function in a 10D (synthetic).
\item a 5D embedded function in a 25D setting (synthetic).
\item A function that has the exact same structure that we propose.
\item SwissFEL data (5D real dimensional domain).
\end{enumerate}

\section{Qualitative evaluation}

\subsection{Finding a matrix when we apply a polynomial kernel onto the input first}
We want to analyse if the gpregression can identify the random matrix, if the input is first put through a polynomial kernel of degree 2. \\

More specifically, the problem looks as follows.
We want to approximate the real function $ f $ through a gaussian process $ g $, with the following condition:

\def\B{
\begin{bmatrix}
    (x - x_0)^2 \\
    (x - x_1)^2
\end{bmatrix}}

\begin{equation}
f \left( W \B \right) \approx g \left( x \right)
\end{equation} 

where $x_0, x_1$ are constants.

\subsection{Finding a matrix when we apply a polynomial kernel onto the input first (feature selection)}
We want to analyse if the gpregression can identify the random matrix, if the input is first put through a polynomial kernel of degree 2. \\

More specifically, the problem looks as follows.
We want to approximate the real function $ f $ through a gaussian process $ g $, with the following condition:

\def\B{
\begin{bmatrix}
    (x - x_0)^2 \\
    (x - x_1)^2
\end{bmatrix}}

\begin{equation}
f \left( W \B \right) \approx g \left( x \right)
\end{equation} 

where $x_0, x_1$ are constants.


\subsection{Subspace identification}


